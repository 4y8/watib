\documentclass[10pt]{article}
\usepackage[utf8]{inputenc}
\usepackage[T1]{fontenc}
\usepackage{minted}
\usepackage[english]{babel}
\usepackage[a4paper, left=1.2in, right=1.2in]{geometry}

\usepackage[backend=bibtex]{biblatex}
\addbibresource{report/report.bib}
\author{Aghilas Y. Boussaa \texttt{<aghilas.boussaa@ens.fr>}}

\title{\textsf{watib}: Design and implementation of an optimising WebAssembly toolchain}
\begin{document}
\maketitle
\begin{abstract}
  We describe watib: a WebAssembly Toolchain written In Bigloo. It has
  been integrated in the Bigloo Scheme compiler and is available as a standalone
  tool. It currently handles, parsing, validation (type checking), optimisation
  and assembly of files in WebAssembly Text format.
\end{abstract}

\section{WebAssembly}
WebAssembly~\cite{haas2017bringing} is an assembly-like language for a
stack-based virtual machine. Its standard~\cite{WebAssemblyCoreSpecification3}
typing rules (validation), semantics, an abstract syntax and two concrete
syntaxes (formats) for the language. The binary format represents WebAssembly
modules as a sequence of bytes. It is taken as input by Wasm virtual machines
such as V8~\cite{V8} or SpiderMonkey~\cite{SpiderMonkey} and manipulated by
toolchains such as binaryen~\cite{Binaryen} or wasm-tools~\cite{WasmTools}. The
textual format represents WebAssembly modules as S-expressions. It is taken as
input by toolchains or written by humans (and compilers such as Bigloo~\cite{Bigloo}).
\begin{figure}[h]
\centering
\begin{tabular}{c c c}
\begin{minipage}{2in}
\begin{verbatim}
(func $add
  (param $x i32)
  (param $y i32)
  (result i32)
  (i32.add (local.get $x) (local.get $y)))
\end{verbatim}
\end{minipage}&
\begin{minipage}{2in}
\begin{verbatim}
  (func $a)
\end{verbatim}
\end{minipage}
&\begin{minipage}{2in}
\begin{verbatim}
  (func $a)
\end{verbatim}
\end{minipage}
\end{tabular}
\caption{Examples of WebAssembly in textual format}
\end{figure}
\section{Validation}
\section{Optimisation}
\printbibliography
\end{document}
